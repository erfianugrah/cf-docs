% Options for packages loaded elsewhere
\PassOptionsToPackage{unicode}{hyperref}
\PassOptionsToPackage{hyphens}{url}
\PassOptionsToPackage{dvipsnames,svgnames,x11names}{xcolor}
%
\documentclass[
]{article}

\usepackage{amsmath,amssymb}
\usepackage{iftex}
\ifPDFTeX
  \usepackage[T1]{fontenc}
  \usepackage[utf8]{inputenc}
  \usepackage{textcomp} % provide euro and other symbols
\else % if luatex or xetex
  \usepackage{unicode-math}
  \defaultfontfeatures{Scale=MatchLowercase}
  \defaultfontfeatures[\rmfamily]{Ligatures=TeX,Scale=1}
\fi
\usepackage{lmodern}
\ifPDFTeX\else  
    % xetex/luatex font selection
  \setmainfont[]{Times New Roman}
  \setmonofont[]{IosevkaTerm Nerd Font Mono}
\fi
% Use upquote if available, for straight quotes in verbatim environments
\IfFileExists{upquote.sty}{\usepackage{upquote}}{}
\IfFileExists{microtype.sty}{% use microtype if available
  \usepackage[]{microtype}
  \UseMicrotypeSet[protrusion]{basicmath} % disable protrusion for tt fonts
}{}
\makeatletter
\@ifundefined{KOMAClassName}{% if non-KOMA class
  \IfFileExists{parskip.sty}{%
    \usepackage{parskip}
  }{% else
    \setlength{\parindent}{0pt}
    \setlength{\parskip}{6pt plus 2pt minus 1pt}}
}{% if KOMA class
  \KOMAoptions{parskip=half}}
\makeatother
\usepackage{xcolor}
\usepackage[top=30mm,left=20mm,heightrounded]{geometry}
\setlength{\emergencystretch}{3em} % prevent overfull lines
\setcounter{secnumdepth}{5}
% Make \paragraph and \subparagraph free-standing
\ifx\paragraph\undefined\else
  \let\oldparagraph\paragraph
  \renewcommand{\paragraph}[1]{\oldparagraph{#1}\mbox{}}
\fi
\ifx\subparagraph\undefined\else
  \let\oldsubparagraph\subparagraph
  \renewcommand{\subparagraph}[1]{\oldsubparagraph{#1}\mbox{}}
\fi

\usepackage{color}
\usepackage{fancyvrb}
\newcommand{\VerbBar}{|}
\newcommand{\VERB}{\Verb[commandchars=\\\{\}]}
\DefineVerbatimEnvironment{Highlighting}{Verbatim}{commandchars=\\\{\}}
% Add ',fontsize=\small' for more characters per line
\usepackage{framed}
\definecolor{shadecolor}{RGB}{248,248,248}
\newenvironment{Shaded}{\begin{snugshade}}{\end{snugshade}}
\newcommand{\AlertTok}[1]{\textcolor[rgb]{0.94,0.16,0.16}{#1}}
\newcommand{\AnnotationTok}[1]{\textcolor[rgb]{0.56,0.35,0.01}{\textbf{\textit{#1}}}}
\newcommand{\AttributeTok}[1]{\textcolor[rgb]{0.13,0.29,0.53}{#1}}
\newcommand{\BaseNTok}[1]{\textcolor[rgb]{0.00,0.00,0.81}{#1}}
\newcommand{\BuiltInTok}[1]{#1}
\newcommand{\CharTok}[1]{\textcolor[rgb]{0.31,0.60,0.02}{#1}}
\newcommand{\CommentTok}[1]{\textcolor[rgb]{0.56,0.35,0.01}{\textit{#1}}}
\newcommand{\CommentVarTok}[1]{\textcolor[rgb]{0.56,0.35,0.01}{\textbf{\textit{#1}}}}
\newcommand{\ConstantTok}[1]{\textcolor[rgb]{0.56,0.35,0.01}{#1}}
\newcommand{\ControlFlowTok}[1]{\textcolor[rgb]{0.13,0.29,0.53}{\textbf{#1}}}
\newcommand{\DataTypeTok}[1]{\textcolor[rgb]{0.13,0.29,0.53}{#1}}
\newcommand{\DecValTok}[1]{\textcolor[rgb]{0.00,0.00,0.81}{#1}}
\newcommand{\DocumentationTok}[1]{\textcolor[rgb]{0.56,0.35,0.01}{\textbf{\textit{#1}}}}
\newcommand{\ErrorTok}[1]{\textcolor[rgb]{0.64,0.00,0.00}{\textbf{#1}}}
\newcommand{\ExtensionTok}[1]{#1}
\newcommand{\FloatTok}[1]{\textcolor[rgb]{0.00,0.00,0.81}{#1}}
\newcommand{\FunctionTok}[1]{\textcolor[rgb]{0.13,0.29,0.53}{\textbf{#1}}}
\newcommand{\ImportTok}[1]{#1}
\newcommand{\InformationTok}[1]{\textcolor[rgb]{0.56,0.35,0.01}{\textbf{\textit{#1}}}}
\newcommand{\KeywordTok}[1]{\textcolor[rgb]{0.13,0.29,0.53}{\textbf{#1}}}
\newcommand{\NormalTok}[1]{#1}
\newcommand{\OperatorTok}[1]{\textcolor[rgb]{0.81,0.36,0.00}{\textbf{#1}}}
\newcommand{\OtherTok}[1]{\textcolor[rgb]{0.56,0.35,0.01}{#1}}
\newcommand{\PreprocessorTok}[1]{\textcolor[rgb]{0.56,0.35,0.01}{\textit{#1}}}
\newcommand{\RegionMarkerTok}[1]{#1}
\newcommand{\SpecialCharTok}[1]{\textcolor[rgb]{0.81,0.36,0.00}{\textbf{#1}}}
\newcommand{\SpecialStringTok}[1]{\textcolor[rgb]{0.31,0.60,0.02}{#1}}
\newcommand{\StringTok}[1]{\textcolor[rgb]{0.31,0.60,0.02}{#1}}
\newcommand{\VariableTok}[1]{\textcolor[rgb]{0.00,0.00,0.00}{#1}}
\newcommand{\VerbatimStringTok}[1]{\textcolor[rgb]{0.31,0.60,0.02}{#1}}
\newcommand{\WarningTok}[1]{\textcolor[rgb]{0.56,0.35,0.01}{\textbf{\textit{#1}}}}

\providecommand{\tightlist}{%
  \setlength{\itemsep}{0pt}\setlength{\parskip}{0pt}}\usepackage{longtable,booktabs,array}
\usepackage{calc} % for calculating minipage widths
% Correct order of tables after \paragraph or \subparagraph
\usepackage{etoolbox}
\makeatletter
\patchcmd\longtable{\par}{\if@noskipsec\mbox{}\fi\par}{}{}
\makeatother
% Allow footnotes in longtable head/foot
\IfFileExists{footnotehyper.sty}{\usepackage{footnotehyper}}{\usepackage{footnote}}
\makesavenoteenv{longtable}
\usepackage{graphicx}
\makeatletter
\def\maxwidth{\ifdim\Gin@nat@width>\linewidth\linewidth\else\Gin@nat@width\fi}
\def\maxheight{\ifdim\Gin@nat@height>\textheight\textheight\else\Gin@nat@height\fi}
\makeatother
% Scale images if necessary, so that they will not overflow the page
% margins by default, and it is still possible to overwrite the defaults
% using explicit options in \includegraphics[width, height, ...]{}
\setkeys{Gin}{width=\maxwidth,height=\maxheight,keepaspectratio}
% Set default figure placement to htbp
\makeatletter
\def\fps@figure{htbp}
\makeatother

\usepackage{fvextra}
\DefineVerbatimEnvironment{Highlighting}{Verbatim}{breaklines,commandchars=\\\{\}}
\DefineVerbatimEnvironment{OutputCode}{Verbatim}{breaklines,commandchars=\\\{\}}
\makeatletter
\@ifpackageloaded{caption}{}{\usepackage{caption}}
\AtBeginDocument{%
\ifdefined\contentsname
  \renewcommand*\contentsname{Table of contents}
\else
  \newcommand\contentsname{Table of contents}
\fi
\ifdefined\listfigurename
  \renewcommand*\listfigurename{List of Figures}
\else
  \newcommand\listfigurename{List of Figures}
\fi
\ifdefined\listtablename
  \renewcommand*\listtablename{List of Tables}
\else
  \newcommand\listtablename{List of Tables}
\fi
\ifdefined\figurename
  \renewcommand*\figurename{Figure}
\else
  \newcommand\figurename{Figure}
\fi
\ifdefined\tablename
  \renewcommand*\tablename{Table}
\else
  \newcommand\tablename{Table}
\fi
}
\@ifpackageloaded{float}{}{\usepackage{float}}
\floatstyle{ruled}
\@ifundefined{c@chapter}{\newfloat{codelisting}{h}{lop}}{\newfloat{codelisting}{h}{lop}[chapter]}
\floatname{codelisting}{Listing}
\newcommand*\listoflistings{\listof{codelisting}{List of Listings}}
\makeatother
\makeatletter
\makeatother
\makeatletter
\@ifpackageloaded{caption}{}{\usepackage{caption}}
\@ifpackageloaded{subcaption}{}{\usepackage{subcaption}}
\makeatother
\ifLuaTeX
  \usepackage{selnolig}  % disable illegal ligatures
\fi
\usepackage{bookmark}

\IfFileExists{xurl.sty}{\usepackage{xurl}}{} % add URL line breaks if available
\urlstyle{same} % disable monospaced font for URLs
\hypersetup{
  pdftitle={How to: Get up and running with VyOS 1.4 with IPsec and GRE},
  pdfauthor={Erfi Anugrah},
  colorlinks=true,
  linkcolor={blue},
  filecolor={Maroon},
  citecolor={Blue},
  urlcolor={Blue},
  pdfcreator={LaTeX via pandoc}}

\title{How to: Get up and running with VyOS 1.4 with IPsec and GRE}
\author{Erfi Anugrah}
\date{2024-05-24}

\begin{document}
\maketitle

\renewcommand*\contentsname{Table of contents}
{
\hypersetup{linkcolor=}
\setcounter{tocdepth}{10}
\tableofcontents
}
\newpage{}

\subsection{Firewall}\label{firewall}

\subsubsection{Global Options}\label{global-options}

\begin{Shaded}
\begin{Highlighting}[numbers=left,,]
\BuiltInTok{set}\NormalTok{ firewall global{-}options all{-}ping }\StringTok{\textquotesingle{}enable\textquotesingle{}}
\BuiltInTok{set}\NormalTok{ firewall global{-}options broadcast{-}ping }\StringTok{\textquotesingle{}disable\textquotesingle{}}
\BuiltInTok{set}\NormalTok{ firewall global{-}options ip{-}src{-}route }\StringTok{\textquotesingle{}disable\textquotesingle{}}
\BuiltInTok{set}\NormalTok{ firewall global{-}options ipv6{-}receive{-}redirects }\StringTok{\textquotesingle{}disable\textquotesingle{}}
\BuiltInTok{set}\NormalTok{ firewall global{-}options ipv6{-}src{-}route }\StringTok{\textquotesingle{}disable\textquotesingle{}}
\BuiltInTok{set}\NormalTok{ firewall global{-}options log{-}martians }\StringTok{\textquotesingle{}enable\textquotesingle{}}
\BuiltInTok{set}\NormalTok{ firewall global{-}options receive{-}redirects }\StringTok{\textquotesingle{}disable\textquotesingle{}}
\BuiltInTok{set}\NormalTok{ firewall global{-}options send{-}redirects }\StringTok{\textquotesingle{}enable\textquotesingle{}}
\BuiltInTok{set}\NormalTok{ firewall global{-}options source{-}validation }\StringTok{\textquotesingle{}disable\textquotesingle{}}
\BuiltInTok{set}\NormalTok{ firewall global{-}options syn{-}cookies }\StringTok{\textquotesingle{}enable\textquotesingle{}}
\end{Highlighting}
\end{Shaded}

\newpage{}

\subsubsection{Network Groups}\label{network-groups}

In this case, I used Cloudflare's IP Ranges:

\begin{Shaded}
\begin{Highlighting}[numbers=left,,]
\BuiltInTok{set}\NormalTok{ firewall group ipv6{-}network{-}group cf{-}ipv6 network }\StringTok{\textquotesingle{}xxxx:xxxx::/32\textquotesingle{}}
\BuiltInTok{set}\NormalTok{ firewall group ipv6{-}network{-}group cf{-}ipv6 network }\StringTok{\textquotesingle{}xxxx:xxxx::/32\textquotesingle{}}
\BuiltInTok{set}\NormalTok{ firewall group ipv6{-}network{-}group cf{-}ipv6 network }\StringTok{\textquotesingle{}xxxx:xxxx::/32\textquotesingle{}}
\BuiltInTok{set}\NormalTok{ firewall group ipv6{-}network{-}group cf{-}ipv6 network }\StringTok{\textquotesingle{}xxxx:xxxx::/32\textquotesingle{}}
\BuiltInTok{set}\NormalTok{ firewall group ipv6{-}network{-}group cf{-}ipv6 network }\StringTok{\textquotesingle{}xxxx:xxxx::/32\textquotesingle{}}
\BuiltInTok{set}\NormalTok{ firewall group ipv6{-}network{-}group cf{-}ipv6 network }\StringTok{\textquotesingle{}xxxx:xxxx::/29\textquotesingle{}}
\BuiltInTok{set}\NormalTok{ firewall group ipv6{-}network{-}group cf{-}ipv6 network }\StringTok{\textquotesingle{}xxxx:xxxx::/32\textquotesingle{}}
\BuiltInTok{set}\NormalTok{ firewall group network{-}group cf{-}ipv4 network }\StringTok{\textquotesingle{}xxx.xxx.48.0/20\textquotesingle{}}
\BuiltInTok{set}\NormalTok{ firewall group network{-}group cf{-}ipv4 network }\StringTok{\textquotesingle{}xxx.xxx.244.0/22\textquotesingle{}}
\BuiltInTok{set}\NormalTok{ firewall group network{-}group cf{-}ipv4 network }\StringTok{\textquotesingle{}xxx.xxx.200.0/22\textquotesingle{}}
\BuiltInTok{set}\NormalTok{ firewall group network{-}group cf{-}ipv4 network }\StringTok{\textquotesingle{}xxx.xxx.4.0/22\textquotesingle{}}
\BuiltInTok{set}\NormalTok{ firewall group network{-}group cf{-}ipv4 network }\StringTok{\textquotesingle{}xxx.xxx.64.0/18\textquotesingle{}}
\BuiltInTok{set}\NormalTok{ firewall group network{-}group cf{-}ipv4 network }\StringTok{\textquotesingle{}xxx.xxx.192.0/18\textquotesingle{}}
\BuiltInTok{set}\NormalTok{ firewall group network{-}group cf{-}ipv4 network }\StringTok{\textquotesingle{}xxx.xxx.240.0/20\textquotesingle{}}
\BuiltInTok{set}\NormalTok{ firewall group network{-}group cf{-}ipv4 network }\StringTok{\textquotesingle{}xxx.xxx.96.0/20\textquotesingle{}}
\BuiltInTok{set}\NormalTok{ firewall group network{-}group cf{-}ipv4 network }\StringTok{\textquotesingle{}xxx.xxx.240.0/22\textquotesingle{}}
\BuiltInTok{set}\NormalTok{ firewall group network{-}group cf{-}ipv4 network }\StringTok{\textquotesingle{}xxx.xxx.128.0/17\textquotesingle{}}
\BuiltInTok{set}\NormalTok{ firewall group network{-}group cf{-}ipv4 network }\StringTok{\textquotesingle{}xxx.xxx.0.0/15\textquotesingle{}}
\BuiltInTok{set}\NormalTok{ firewall group network{-}group cf{-}ipv4 network }\StringTok{\textquotesingle{}xxx.xxx.0.0/13\textquotesingle{}}
\BuiltInTok{set}\NormalTok{ firewall group network{-}group cf{-}ipv4 network }\StringTok{\textquotesingle{}xxx.xxx.0.0/14\textquotesingle{}}
\BuiltInTok{set}\NormalTok{ firewall group network{-}group cf{-}ipv4 network }\StringTok{\textquotesingle{}xxx.xxx.0.0/13\textquotesingle{}}
\BuiltInTok{set}\NormalTok{ firewall group network{-}group cf{-}ipv4 network }\StringTok{\textquotesingle{}xxx.xxx.72.0/22\textquotesingle{}}
\end{Highlighting}
\end{Shaded}

\newpage{}

\subsubsection{Jump Filters}\label{jump-filters}

This is based on the Netfilter project.

Input filter is for the WAN, destination being the router which is on
the prerouting stage. Forward filter is for the inbound-interfaces,
which is into the postrouting and egress stages.

\begin{Shaded}
\begin{Highlighting}[numbers=left,,]
\BuiltInTok{set}\NormalTok{ firewall ipv4 forward filter default{-}action }\StringTok{\textquotesingle{}accept\textquotesingle{}}
\BuiltInTok{set}\NormalTok{ firewall ipv4 forward filter rule 5 action }\StringTok{\textquotesingle{}jump\textquotesingle{}}
\BuiltInTok{set}\NormalTok{ firewall ipv4 forward filter rule 5 inbound{-}interface name }\StringTok{\textquotesingle{}pppoe0\textquotesingle{}}
\BuiltInTok{set}\NormalTok{ firewall ipv4 forward filter rule 5 jump{-}target }\StringTok{\textquotesingle{}EXTERNAL{-}IN\textquotesingle{}}
\BuiltInTok{set}\NormalTok{ firewall ipv4 forward filter rule 10 action }\StringTok{\textquotesingle{}jump\textquotesingle{}}
\BuiltInTok{set}\NormalTok{ firewall ipv4 forward filter rule 10 inbound{-}interface name }\StringTok{\textquotesingle{}eth1\textquotesingle{}}
\BuiltInTok{set}\NormalTok{ firewall ipv4 forward filter rule 10 jump{-}target }\StringTok{\textquotesingle{}INTERNAL1\textquotesingle{}}
\BuiltInTok{set}\NormalTok{ firewall ipv4 forward filter rule 20 action }\StringTok{\textquotesingle{}jump\textquotesingle{}}
\BuiltInTok{set}\NormalTok{ firewall ipv4 forward filter rule 20 inbound{-}interface name }\StringTok{\textquotesingle{}eth1\textquotesingle{}}
\BuiltInTok{set}\NormalTok{ firewall ipv4 forward filter rule 20 jump{-}target }\StringTok{\textquotesingle{}INTERNAL2\textquotesingle{}}
\BuiltInTok{set}\NormalTok{ firewall ipv4 input filter default{-}action }\StringTok{\textquotesingle{}accept\textquotesingle{}}
\BuiltInTok{set}\NormalTok{ firewall ipv4 input filter rule 5 action }\StringTok{\textquotesingle{}jump\textquotesingle{}}
\BuiltInTok{set}\NormalTok{ firewall ipv4 input filter rule 5 inbound{-}interface name }\StringTok{\textquotesingle{}pppoe0\textquotesingle{}}
\BuiltInTok{set}\NormalTok{ firewall ipv4 input filter rule 5 jump{-}target }\StringTok{\textquotesingle{}EXTERNAL{-}LOCAL\textquotesingle{}}
\end{Highlighting}
\end{Shaded}

\newpage{}

\subsubsection{Firewall rules for the input
filter}\label{firewall-rules-for-the-input-filter}

Since we only care about the tunnels in this case, we will be focusing
on the input filter rules here:

\begin{Shaded}
\begin{Highlighting}[numbers=left,,]
\BuiltInTok{set}\NormalTok{ firewall ipv4 name EXTERNAL{-}LOCAL default{-}action }\StringTok{\textquotesingle{}drop\textquotesingle{}}
\BuiltInTok{set}\NormalTok{ firewall ipv4 name EXTERNAL{-}LOCAL default{-}log}
\BuiltInTok{set}\NormalTok{ firewall ipv4 name EXTERNAL{-}LOCAL rule 10 action }\StringTok{\textquotesingle{}accept\textquotesingle{}}
\BuiltInTok{set}\NormalTok{ firewall ipv4 name EXTERNAL{-}LOCAL rule 10 log}
\BuiltInTok{set}\NormalTok{ firewall ipv4 name EXTERNAL{-}LOCAL rule 10 state }\StringTok{\textquotesingle{}established\textquotesingle{}}
\BuiltInTok{set}\NormalTok{ firewall ipv4 name EXTERNAL{-}LOCAL rule 10 state }\StringTok{\textquotesingle{}related\textquotesingle{}}
\BuiltInTok{set}\NormalTok{ firewall ipv4 name EXTERNAL{-}LOCAL rule 20 action }\StringTok{\textquotesingle{}accept\textquotesingle{}}
\BuiltInTok{set}\NormalTok{ firewall ipv4 name EXTERNAL{-}LOCAL rule 20 log}
\BuiltInTok{set}\NormalTok{ firewall ipv4 name EXTERNAL{-}LOCAL rule 20 protocol }\StringTok{\textquotesingle{}icmp\textquotesingle{}}
\BuiltInTok{set}\NormalTok{ firewall ipv4 name EXTERNAL{-}LOCAL rule 40 action }\StringTok{\textquotesingle{}accept\textquotesingle{}}
\BuiltInTok{set}\NormalTok{ firewall ipv4 name EXTERNAL{-}LOCAL rule 40 description }\StringTok{\textquotesingle{}magic{-}wan\textquotesingle{}}
\BuiltInTok{set}\NormalTok{ firewall ipv4 name EXTERNAL{-}LOCAL rule 40 log}
\BuiltInTok{set}\NormalTok{ firewall ipv4 name EXTERNAL{-}LOCAL rule 40 protocol }\StringTok{\textquotesingle{}gre\textquotesingle{}}
\BuiltInTok{set}\NormalTok{ firewall ipv4 name EXTERNAL{-}LOCAL rule 40 source group network{-}group }\StringTok{\textquotesingle{}cf{-}ipv4\textquotesingle{}}
\BuiltInTok{set}\NormalTok{ firewall ipv4 name EXTERNAL{-}LOCAL rule 50 action }\StringTok{\textquotesingle{}accept\textquotesingle{}}
\BuiltInTok{set}\NormalTok{ firewall ipv4 name EXTERNAL{-}LOCAL rule 50 description }\StringTok{\textquotesingle{}magic{-}wan{-}ipsec\textquotesingle{}}
\BuiltInTok{set}\NormalTok{ firewall ipv4 name EXTERNAL{-}LOCAL rule 50 log}
\BuiltInTok{set}\NormalTok{ firewall ipv4 name EXTERNAL{-}LOCAL rule 50 protocol }\StringTok{\textquotesingle{}esp\textquotesingle{}}
\BuiltInTok{set}\NormalTok{ firewall ipv4 name EXTERNAL{-}LOCAL rule 50 source group network{-}group }\StringTok{\textquotesingle{}cf{-}ipv4\textquotesingle{}}
\BuiltInTok{set}\NormalTok{ firewall ipv4 name EXTERNAL{-}LOCAL rule 51 action }\StringTok{\textquotesingle{}accept\textquotesingle{}}
\BuiltInTok{set}\NormalTok{ firewall ipv4 name EXTERNAL{-}LOCAL rule 51 description }\StringTok{\textquotesingle{}magic{-}wan{-}ipsec\textquotesingle{}}
\BuiltInTok{set}\NormalTok{ firewall ipv4 name EXTERNAL{-}LOCAL rule 51 destination port }\StringTok{\textquotesingle{}500\textquotesingle{}}
\BuiltInTok{set}\NormalTok{ firewall ipv4 name EXTERNAL{-}LOCAL rule 51 log}
\BuiltInTok{set}\NormalTok{ firewall ipv4 name EXTERNAL{-}LOCAL rule 51 protocol }\StringTok{\textquotesingle{}udp\textquotesingle{}}
\BuiltInTok{set}\NormalTok{ firewall ipv4 name EXTERNAL{-}LOCAL rule 51 source group network{-}group }\StringTok{\textquotesingle{}cf{-}ipv4\textquotesingle{}}
\BuiltInTok{set}\NormalTok{ firewall ipv4 name EXTERNAL{-}LOCAL rule 52 action }\StringTok{\textquotesingle{}accept\textquotesingle{}}
\BuiltInTok{set}\NormalTok{ firewall ipv4 name EXTERNAL{-}LOCAL rule 52 description }\StringTok{\textquotesingle{}magic{-}wan{-}ipsec\textquotesingle{}}
\BuiltInTok{set}\NormalTok{ firewall ipv4 name EXTERNAL{-}LOCAL rule 52 destination port }\StringTok{\textquotesingle{}4500\textquotesingle{}}
\BuiltInTok{set}\NormalTok{ firewall ipv4 name EXTERNAL{-}LOCAL rule 52 log}
\BuiltInTok{set}\NormalTok{ firewall ipv4 name EXTERNAL{-}LOCAL rule 52 protocol }\StringTok{\textquotesingle{}udp\textquotesingle{}}
\BuiltInTok{set}\NormalTok{ firewall ipv4 name EXTERNAL{-}LOCAL rule 52 source group network{-}group }\StringTok{\textquotesingle{}cf{-}ipv4\textquotesingle{}}
\BuiltInTok{set}\NormalTok{ firewall ipv4 name EXTERNAL{-}LOCAL rule 61 action }\StringTok{\textquotesingle{}accept\textquotesingle{}}
\BuiltInTok{set}\NormalTok{ firewall ipv4 name EXTERNAL{-}LOCAL rule 61 description }\StringTok{\textquotesingle{}sflow\textquotesingle{}}
\BuiltInTok{set}\NormalTok{ firewall ipv4 name EXTERNAL{-}LOCAL rule 61 destination port }\StringTok{\textquotesingle{}6343\textquotesingle{}}
\BuiltInTok{set}\NormalTok{ firewall ipv4 name EXTERNAL{-}LOCAL rule 61 log}
\BuiltInTok{set}\NormalTok{ firewall ipv4 name EXTERNAL{-}LOCAL rule 61 protocol }\StringTok{\textquotesingle{}tcp\_udp\textquotesingle{}}
\BuiltInTok{set}\NormalTok{ firewall ipv4 name EXTERNAL{-}LOCAL rule 61 source group network{-}group }\StringTok{\textquotesingle{}cf{-}ipv4\textquotesingle{}}
\end{Highlighting}
\end{Shaded}

\newpage{}

\subsection{WAN}\label{wan}

In my case since it is a PPPoE connection to the ISP, I will set a VLAN
tag on the ethernet interface that I designated for WAN:

\begin{Shaded}
\begin{Highlighting}[numbers=left,,]
\BuiltInTok{set}\NormalTok{ interfaces ethernet eth0 description }\StringTok{\textquotesingle{}EXTERNAL\textquotesingle{}}
\BuiltInTok{set}\NormalTok{ interfaces ethernet eth0 duplex }\StringTok{\textquotesingle{}auto\textquotesingle{}}
\BuiltInTok{set}\NormalTok{ interfaces ethernet eth0 hw{-}id }\StringTok{\textquotesingle{}xx:xx:xx:xx:xx:71\textquotesingle{}}
\BuiltInTok{set}\NormalTok{ interfaces ethernet eth0 ip disable{-}arp{-}filter}
\BuiltInTok{set}\NormalTok{ interfaces ethernet eth0 offload gro}
\BuiltInTok{set}\NormalTok{ interfaces ethernet eth0 offload gso}
\BuiltInTok{set}\NormalTok{ interfaces ethernet eth0 offload rps}
\BuiltInTok{set}\NormalTok{ interfaces ethernet eth0 offload sg}
\BuiltInTok{set}\NormalTok{ interfaces ethernet eth0 offload tso}
\BuiltInTok{set}\NormalTok{ interfaces ethernet eth0 speed }\StringTok{\textquotesingle{}auto\textquotesingle{}}
\BuiltInTok{set}\NormalTok{ interfaces ethernet eth0 vif 6 ip disable{-}arp{-}filter}
\end{Highlighting}
\end{Shaded}

This would then be used to create the PPPoE interface, do note the TCP
clamping (if) required by your ISP:

\begin{Shaded}
\begin{Highlighting}[numbers=left,,]
\BuiltInTok{set}\NormalTok{ interfaces pppoe pppoe0 authentication password xxxxxx}
\BuiltInTok{set}\NormalTok{ interfaces pppoe pppoe0 authentication username xxxxxx}
\BuiltInTok{set}\NormalTok{ interfaces pppoe pppoe0 description }\StringTok{\textquotesingle{}kpn\textquotesingle{}}
\BuiltInTok{set}\NormalTok{ interfaces pppoe pppoe0 ip adjust{-}mss }\StringTok{\textquotesingle{}clamp{-}mss{-}to{-}pmtu\textquotesingle{}}
\BuiltInTok{set}\NormalTok{ interfaces pppoe pppoe0 no{-}peer{-}dns}
\BuiltInTok{set}\NormalTok{ interfaces pppoe pppoe0 source{-}interface }\StringTok{\textquotesingle{}eth0.6\textquotesingle{}}
\end{Highlighting}
\end{Shaded}

\newpage{}

\subsection{NAT}\label{nat}

Since we are focusing on the tunnels, we are just gonna set
the~masquerade rules (later on you can setup the other ethernet
interfaces:

\begin{Shaded}
\begin{Highlighting}[numbers=left,,]
\BuiltInTok{set}\NormalTok{ nat source rule 20 description }\StringTok{\textquotesingle{}pppoe\textquotesingle{}}
\BuiltInTok{set}\NormalTok{ nat source rule 20 log}
\BuiltInTok{set}\NormalTok{ nat source rule 20 outbound{-}interface name }\StringTok{\textquotesingle{}pppoe0\textquotesingle{}}
\BuiltInTok{set}\NormalTok{ nat source rule 20 source address }\StringTok{\textquotesingle{}xxx.xxx.0.0/16\textquotesingle{}}
\BuiltInTok{set}\NormalTok{ nat source rule 20 translation address }\StringTok{\textquotesingle{}masquerade\textquotesingle{}}
\end{Highlighting}
\end{Shaded}

\subsection{GRE}\label{gre}

\texttt{tun0} interface, not the MSS clamping from the GRE overhead:

\begin{Shaded}
\begin{Highlighting}[numbers=left,,]
\BuiltInTok{set}\NormalTok{ interfaces tunnel tun0 address }\StringTok{\textquotesingle{}xxx.xxx.99.20/31\textquotesingle{}}
\BuiltInTok{set}\NormalTok{ interfaces tunnel tun0 description }\StringTok{\textquotesingle{}magic{-}wan\textquotesingle{}}
\BuiltInTok{set}\NormalTok{ interfaces tunnel tun0 encapsulation }\StringTok{\textquotesingle{}gre\textquotesingle{}}
\BuiltInTok{set}\NormalTok{ interfaces tunnel tun0 ip adjust{-}mss }\StringTok{\textquotesingle{}1436\textquotesingle{}}
\BuiltInTok{set}\NormalTok{ interfaces tunnel tun0 ip disable{-}arp{-}filter}
\BuiltInTok{set}\NormalTok{ interfaces tunnel tun0 remote }\StringTok{\textquotesingle{}xxx.xxx.66.5\textquotesingle{}} \CommentTok{\# Cloudflare\textquotesingle{}s endpoint}
\BuiltInTok{set}\NormalTok{ interfaces tunnel tun0 source{-}address }\StringTok{\textquotesingle{}xxx.xxx.81.4.2\textquotesingle{}} \CommentTok{\# Your IP}
\end{Highlighting}
\end{Shaded}

\subsection{IPsec}\label{ipsec}

\texttt{vti0} interface, note the MSS clamping from the IPsec overhead:

\begin{Shaded}
\begin{Highlighting}[numbers=left,,]
\BuiltInTok{set}\NormalTok{ interfaces vti vti0 address }\StringTok{\textquotesingle{}xxx.xxx.100.20/31\textquotesingle{}}
\BuiltInTok{set}\NormalTok{ interfaces vti vti0 description }\StringTok{\textquotesingle{}magic{-}wan{-}ipsec\textquotesingle{}}
\BuiltInTok{set}\NormalTok{ interfaces vti vti0 ip adjust{-}mss }\StringTok{\textquotesingle{}1350\textquotesingle{}}
\BuiltInTok{set}\NormalTok{ interfaces vti vti0 ip disable{-}arp{-}filter}
\end{Highlighting}
\end{Shaded}

\newpage{}

\subsubsection{VPN}\label{vpn}

This is the site-to-site VPN setup that will be using the
\texttt{vti0}~we have initially setup to initiate the IPsec tunnels to
Cloudflare:

\begin{Shaded}
\begin{Highlighting}[numbers=left,,]
\BuiltInTok{set}\NormalTok{ vpn ipsec authentication psk cf{-}ipsec id }\StringTok{\textquotesingle{}xxx.xxx.242.5\textquotesingle{}} \CommentTok{\# Cloudflare\textquotesingle{}s endpoint}
\BuiltInTok{set}\NormalTok{ vpn ipsec authentication psk cf{-}ipsec secret xxxxxx }\CommentTok{\# Secret token that you generated for use or randomly via the Tunnel creation with Cloudflare}
\BuiltInTok{set}\NormalTok{ vpn ipsec esp{-}group vyos{-}nl{-}esp lifetime }\StringTok{\textquotesingle{}14400\textquotesingle{}}
\BuiltInTok{set}\NormalTok{ vpn ipsec esp{-}group vyos{-}nl{-}esp mode }\StringTok{\textquotesingle{}tunnel\textquotesingle{}}
\BuiltInTok{set}\NormalTok{ vpn ipsec esp{-}group vyos{-}nl{-}esp pfs }\StringTok{\textquotesingle{}enable\textquotesingle{}}
\BuiltInTok{set}\NormalTok{ vpn ipsec esp{-}group vyos{-}nl{-}esp proposal 1 encryption }\StringTok{\textquotesingle{}aes256gcm128\textquotesingle{}}
\BuiltInTok{set}\NormalTok{ vpn ipsec esp{-}group vyos{-}nl{-}esp proposal 1 hash }\StringTok{\textquotesingle{}sha512\textquotesingle{}}
\BuiltInTok{set}\NormalTok{ vpn ipsec ike{-}group vyos{-}nl{-}ike close{-}action }\StringTok{\textquotesingle{}start\textquotesingle{}}
\BuiltInTok{set}\NormalTok{ vpn ipsec ike{-}group vyos{-}nl{-}ike dead{-}peer{-}detection action }\StringTok{\textquotesingle{}restart\textquotesingle{}}
\BuiltInTok{set}\NormalTok{ vpn ipsec ike{-}group vyos{-}nl{-}ike dead{-}peer{-}detection interval }\StringTok{\textquotesingle{}30\textquotesingle{}}
\BuiltInTok{set}\NormalTok{ vpn ipsec ike{-}group vyos{-}nl{-}ike dead{-}peer{-}detection timeout }\StringTok{\textquotesingle{}120\textquotesingle{}}
\BuiltInTok{set}\NormalTok{ vpn ipsec ike{-}group vyos{-}nl{-}ike disable{-}mobike}
\BuiltInTok{set}\NormalTok{ vpn ipsec ike{-}group vyos{-}nl{-}ike key{-}exchange }\StringTok{\textquotesingle{}ikev2\textquotesingle{}}
\BuiltInTok{set}\NormalTok{ vpn ipsec ike{-}group vyos{-}nl{-}ike lifetime }\StringTok{\textquotesingle{}14400\textquotesingle{}}
\BuiltInTok{set}\NormalTok{ vpn ipsec ike{-}group vyos{-}nl{-}ike proposal 1 dh{-}group }\StringTok{\textquotesingle{}14\textquotesingle{}}
\BuiltInTok{set}\NormalTok{ vpn ipsec ike{-}group vyos{-}nl{-}ike proposal 1 encryption }\StringTok{\textquotesingle{}aes256gcm128\textquotesingle{}}
\BuiltInTok{set}\NormalTok{ vpn ipsec ike{-}group vyos{-}nl{-}ike proposal 1 hash }\StringTok{\textquotesingle{}sha512\textquotesingle{}}
\BuiltInTok{set}\NormalTok{ vpn ipsec interface }\StringTok{\textquotesingle{}pppoe0\textquotesingle{}}
\BuiltInTok{set}\NormalTok{ vpn ipsec log level }\StringTok{\textquotesingle{}2\textquotesingle{}}
\BuiltInTok{set}\NormalTok{ vpn ipsec log subsystem }\StringTok{\textquotesingle{}any\textquotesingle{}}
\BuiltInTok{set}\NormalTok{ vpn ipsec options disable{-}route{-}autoinstall}
\BuiltInTok{set}\NormalTok{ vpn ipsec site{-}to{-}site peer magic{-}wan{-}ipsec authentication local{-}id }\StringTok{\textquotesingle{}ID\textquotesingle{}} \CommentTok{\# The authentication ID you can get from the API}
\BuiltInTok{set}\NormalTok{ vpn ipsec site{-}to{-}site peer magic{-}wan{-}ipsec authentication mode }\StringTok{\textquotesingle{}pre{-}shared{-}secret\textquotesingle{}}
\BuiltInTok{set}\NormalTok{ vpn ipsec site{-}to{-}site peer magic{-}wan{-}ipsec authentication remote{-}id }\StringTok{\textquotesingle{}xxx.xxx.242.5\textquotesingle{}} \CommentTok{\# Cloudflare\textquotesingle{}s endpoint}
\BuiltInTok{set}\NormalTok{ vpn ipsec site{-}to{-}site peer magic{-}wan{-}ipsec connection{-}type }\StringTok{\textquotesingle{}initiate\textquotesingle{}}
\BuiltInTok{set}\NormalTok{ vpn ipsec site{-}to{-}site peer magic{-}wan{-}ipsec ike{-}group }\StringTok{\textquotesingle{}vyos{-}nl{-}ike\textquotesingle{}}
\BuiltInTok{set}\NormalTok{ vpn ipsec site{-}to{-}site peer magic{-}wan{-}ipsec ikev2{-}reauth }\StringTok{\textquotesingle{}yes\textquotesingle{}}
\BuiltInTok{set}\NormalTok{ vpn ipsec site{-}to{-}site peer magic{-}wan{-}ipsec local{-}address }\StringTok{\textquotesingle{}xxx.xxx.81.42\textquotesingle{}} \CommentTok{\# The IP assigned by your ISP}
\BuiltInTok{set}\NormalTok{ vpn ipsec site{-}to{-}site peer magic{-}wan{-}ipsec remote{-}address }\StringTok{\textquotesingle{}xxx.xxx.242.5\textquotesingle{}} \CommentTok{\# Cloudflare\textquotesingle{}s endpoint}
\BuiltInTok{set}\NormalTok{ vpn ipsec site{-}to{-}site peer magic{-}wan{-}ipsec vti bind }\StringTok{\textquotesingle{}vti0\textquotesingle{}}
\BuiltInTok{set}\NormalTok{ vpn ipsec site{-}to{-}site peer magic{-}wan{-}ipsec vti esp{-}group }\StringTok{\textquotesingle{}vyos{-}nl{-}esp\textquotesingle{}}
\end{Highlighting}
\end{Shaded}

\subsubsection{Replay Window}\label{replay-window}

Refer to the
\href{https://developers.cloudflare.com/magic-wan/reference/anti-replay-protection/\#1-and-anti-replay-protection}{docs}
when deciding to set it to \texttt{0} or not:

\begin{Shaded}
\begin{Highlighting}[numbers=left,,]
\BuiltInTok{set}\NormalTok{ vpn ipsec site{-}to{-}site peer magic{-}wan{-}ipsec replay{-}window }\StringTok{\textquotesingle{}0\textquotesingle{}}
\end{Highlighting}
\end{Shaded}

\newpage{}

\subsubsection{Strongswan}\label{strongswan}

The configuration above essentially updates a script that generates the
Strongwan config below, this can be found in
\texttt{/etc/swanctl/swanctl.conf}:

\begin{Shaded}
\begin{Highlighting}[numbers=left,,]
\ExtensionTok{connections}\NormalTok{ \{}
    \ExtensionTok{magic{-}wan{-}ipsec}\NormalTok{ \{}
        \ExtensionTok{proposals}\NormalTok{ = aes256gcm128{-}sha512{-}modp2048}
        \ExtensionTok{version}\NormalTok{ = 2}
        \ExtensionTok{local\_addrs}\NormalTok{ = xxx.xxx.xxx.42 }\CommentTok{\# Your IP address}
        \ExtensionTok{remote\_addrs}\NormalTok{ = xxx.xxx.xxx.5 }\CommentTok{\# Cloudflare\textquotesingle{}s endpoint}
        \ExtensionTok{dpd\_timeout}\NormalTok{ = 120}
        \ExtensionTok{dpd\_delay}\NormalTok{ = 30}
        \ExtensionTok{rekey\_time}\NormalTok{ = 14400s}
        \ExtensionTok{mobike}\NormalTok{ = no}
        \ExtensionTok{keyingtries}\NormalTok{ = 0}
        \BuiltInTok{local}\NormalTok{ \{}
            \FunctionTok{id}\NormalTok{ = }\StringTok{"ID"} \CommentTok{\# From Cloudfare IPsec Tunnel API}
            \ExtensionTok{auth}\NormalTok{ = psk}
        \ErrorTok{\}}
        \ExtensionTok{remote}\NormalTok{ \{}
            \FunctionTok{id}\NormalTok{ = }\StringTok{"x.x.242.5"}
            \ExtensionTok{auth}\NormalTok{ = psk}
        \ErrorTok{\}}
        \ExtensionTok{children}\NormalTok{ \{}
            \ExtensionTok{magic{-}wan{-}ipsec{-}vti}\NormalTok{ \{}
                \ExtensionTok{esp\_proposals}\NormalTok{ = aes256gcm128{-}sha512{-}modp2048}
                \ExtensionTok{life\_time}\NormalTok{ = 14400s}
                \ExtensionTok{local\_ts}\NormalTok{ = 0.0.0.0/0,::/0}
                \ExtensionTok{remote\_ts}\NormalTok{ = 0.0.0.0/0,::/0}
                \ExtensionTok{updown}\NormalTok{ = }\StringTok{"/etc/ipsec.d/vti{-}up{-}down vti0"}
                \ExtensionTok{if\_id\_in}\NormalTok{ = 1}
                \ExtensionTok{if\_id\_out}\NormalTok{ = 1}
                \ExtensionTok{ipcomp}\NormalTok{ = no}
                \ExtensionTok{mode}\NormalTok{ = tunnel}
                \ExtensionTok{start\_action}\NormalTok{ = start}
                \ExtensionTok{dpd\_action}\NormalTok{ = restart}
                \ExtensionTok{close\_action}\NormalTok{ = start}
                \ExtensionTok{replay\_window}\NormalTok{ = 0}
            \ErrorTok{\}}
        \ErrorTok{\}}
    \ErrorTok{\}}

\ErrorTok{\}}

\ExtensionTok{pools}\NormalTok{ \{}
\ErrorTok{\}}

\ExtensionTok{secrets}\NormalTok{ \{}
    \ExtensionTok{ike{-}cf{-}ipsec}\NormalTok{ \{}
        \CommentTok{\# ID\textquotesingle{}s from auth psk \textless{}tag\textgreater{} id xxx}
        \ExtensionTok{id{-}xxxxxxx}\NormalTok{ = }\StringTok{"x.x.242.5"} \CommentTok{\# Cloudflare\textquotesingle{}s endpoint}
        \ExtensionTok{secret}\NormalTok{ = }\StringTok{"SECRET"} \CommentTok{\# Secret used to create tunnel}
    \ErrorTok{\}}

\ErrorTok{\}}
\end{Highlighting}
\end{Shaded}

\subsubsection{Bidirectional
Health-checks}\label{bidirectional-health-checks}

As mentioned above, without replay-window being zero, you can't get
health-checks working unless you select the option when creating the
tunnel on Cloudflare.

\subsection{Policy Based Routing}\label{policy-based-routing}

We set the routing tables:

\begin{Shaded}
\begin{Highlighting}[numbers=left,,]
\BuiltInTok{set}\NormalTok{ protocols static table 10 route xxx.xxx.0.0/0 interface tun0}
\BuiltInTok{set}\NormalTok{ protocols static table 20 route xxx.xxx.0.0/0 interface vti0}
\end{Highlighting}
\end{Shaded}

And in this case, I choose to route selectively via the PBRs below:

\subsubsection{GRE}\label{gre-1}

\begin{Shaded}
\begin{Highlighting}[numbers=left,,]
\BuiltInTok{set}\NormalTok{ policy route magic{-}wan{-}gre{-}rasp rule 5 description }\StringTok{\textquotesingle{}magic{-}wan{-}gre{-}rasp\textquotesingle{}}
\BuiltInTok{set}\NormalTok{ policy route magic{-}wan{-}gre{-}rasp{-}tcp{-}udp rule 5 protocol }\StringTok{\textquotesingle{}tcp\_udp\textquotesingle{}}
\BuiltInTok{set}\NormalTok{ policy route magic{-}wan{-}gre{-}rasp rule 5 log}
\BuiltInTok{set}\NormalTok{ policy route magic{-}wan{-}gre{-}rasp rule 5 set table }\StringTok{\textquotesingle{}10\textquotesingle{}}
\BuiltInTok{set}\NormalTok{ policy route magic{-}wan{-}gre{-}rasp rule 5 source address }\StringTok{\textquotesingle{}xxx.xxx.69.7\textquotesingle{}}
\end{Highlighting}
\end{Shaded}

\subsubsection{IPsec}\label{ipsec-1}

\begin{Shaded}
\begin{Highlighting}[numbers=left,,]
\BuiltInTok{set}\NormalTok{ policy route magic{-}wan{-}ipsec{-}rasp{-}tcp{-}udp rule 5 log}
\BuiltInTok{set}\NormalTok{ policy route magic{-}wan{-}ipsec{-}rasp rule 5 description }\StringTok{\textquotesingle{}magic{-}wan{-}ipsec{-}rasp\textquotesingle{}}
\BuiltInTok{set}\NormalTok{ policy route magic{-}wan{-}ipsec{-}rasp{-}tcp{-}udp rule 5 protocol }\StringTok{\textquotesingle{}tcp\_udp\textquotesingle{}}
\BuiltInTok{set}\NormalTok{ policy route magic{-}wan{-}ipsec{-}rasp{-}tcp{-}udp rule 5 set table }\StringTok{\textquotesingle{}20\textquotesingle{}}
\BuiltInTok{set}\NormalTok{ policy route magic{-}wan{-}ipsec{-}rasp{-}tcp{-}udp rule 5 source address }\StringTok{\textquotesingle{}xxx.xxx.69.7\textquotesingle{}}
\end{Highlighting}
\end{Shaded}

\subsection{NFT Rulesets}\label{nft-rulesets}

The configuration rules above just create \texttt{nft} rules and if you
did decide to, it's best to create your own ruleset, as the config will
override.

Run \texttt{sudo\ nft\ list\ ruleset} to see the current rules, this
would include firewall, NAT, PBRs etc.

\newpage{}

\subsection{Sysctl}\label{sysctl}

If required the following might need to be set:

\begin{Shaded}
\begin{Highlighting}[numbers=left,,]
\BuiltInTok{set}\NormalTok{ system sysctl parameter net.core.rmem\_max value }\StringTok{\textquotesingle{}2500000\textquotesingle{}}
\BuiltInTok{set}\NormalTok{ system sysctl parameter net.core.wmem\_max value }\StringTok{\textquotesingle{}2500000\textquotesingle{}}
\BuiltInTok{set}\NormalTok{ system sysctl parameter net.ipv4.conf.all.accept\_local value }\StringTok{\textquotesingle{}1\textquotesingle{}}
\BuiltInTok{set}\NormalTok{ system sysctl parameter net.ipv4.conf.all.accept\_redirects value }\StringTok{\textquotesingle{}0\textquotesingle{}}
\BuiltInTok{set}\NormalTok{ system sysctl parameter net.ipv4.conf.all.arp\_filter value }\StringTok{\textquotesingle{}0\textquotesingle{}}
\BuiltInTok{set}\NormalTok{ system sysctl parameter net.ipv4.conf.all.rp\_filter value }\StringTok{\textquotesingle{}0\textquotesingle{}}
\BuiltInTok{set}\NormalTok{ system sysctl parameter net.ipv4.conf.all.send\_redirects value }\StringTok{\textquotesingle{}0\textquotesingle{}}
\BuiltInTok{set}\NormalTok{ system sysctl parameter net.ipv4.conf.default.arp\_filter value }\StringTok{\textquotesingle{}0\textquotesingle{}}
\BuiltInTok{set}\NormalTok{ system sysctl parameter net.ipv4.conf.default.rp\_filter value }\StringTok{\textquotesingle{}0\textquotesingle{}}
\end{Highlighting}
\end{Shaded}

This has to do with \texttt{reverse\ path} filtering,
\texttt{ip\ forwarding} and \texttt{rmem} and \texttt{wmem} for
\href{https://github.com/quic-go/quic-go/wiki/UDP-Buffer-Sizes}{QUIC}
tunnels should you choose to set up Cloudflare Tunnels on the same
machine. Take a look at the definitions on
\href{https://sysctl-explorer.net/}{sysctl-explorer}.



\end{document}
